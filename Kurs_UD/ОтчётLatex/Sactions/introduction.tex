\newpage
\begin{center}
    \section*{ВВЕДЕНИЕ}
        \addcontentsline{toc}{section}{ВВЕДЕНИЕ}  
\end{center}

Люди всегда разрабатывают, изобретают что-либо новое. Одним из выдающихся изобретений человека является самолет. С помощью самолетов и других летательных аппаратов люди могут быстро перемещаться из одной точки пространства в другую, перевозить грузы, также использовать ЛА и в качестве оружия. Но управление летательным аппаратом-непростая задача. И для того чтобы облегчить пилотирование самолета, снизить загрузку летчика используются системы автоматического управления. Поэтому такая тема курсовой работы как «Система стабилизации курса накренением самолета» всегда будет актуальной.  
Основанием для выполнения данной работы было задание, выданное преподавателем. Цели данной работы –проведение синтеза контуров САУ и анализ системы после выполнения синтеза, также закрепление знаний по дисциплине «Управление движением».

Решение такой задачи как «моделирование системы стабилизации курса накренением самолета» в наше время не является новой, работы с такими системами всегда проводятся и будут проводиться специалистами по управлению движением самолетов. Хотя моделирование и расчеты, которые необходимо провести в данной курсовой работе, не являются очень сложными по сравнению с другими задачами, решаемыми в управлении движением, выполнение работ данного типа всегда необходимо проводить для получения практических навыков выполнения расчетов, повторения и более глубокого изучения материала. Данная работа может быть использована далее в качестве примера выполнения синтеза системы стабилизации курса накренением самолета. 

В настоящее время, так как изобретено большое количество видов самолетов и под каждый самолет и возможности летчика необходимо подстраивать систему управления, то данная наука никогда не будет «стоять на месте» и исследования, связанные с управлением движением, будут актуальны и будут всегда проводиться.
