\newpage
\begin{center}
    \textbf{Реферат}
\end{center}

Курсовая работа по дисциплине «Управление движением ЛА» «Система стабилизации вертикальной скорости самолёта» 38с., 13 таблиц, 23 иллюстрации, 4 источника информации.

ВЫБОР ПАРАМЕТРОВ ПРИВОДА, РАСЧЕТ КОЭФФИЦИЕНТОВ ОБРАТНЫХ СВЯЗЕЙ И КОЭФФИЦИЕНТОВ СТАБИЛИЗАЦИИ, МОДЕЛИРОВАНИЕ ЛИНЕЙНОЙ САУ, МОДЕЛИРОВАНИЕ НЕЛИНЕЙНОЙ САУ

Цель работы-расчет коэффициентов и моделирование системы стабилизации вертикальной скорости самолета для Concorde: выбор параметров привода, расчет и оценка коэффициентов обратных связей и коэффициентов стабилизации системы, частотный анализ контуров системы, моделирование и анализ линейной и нелинейной САУ. 

Исходные данные были получены из альбома исходных данных кафедры для различных самолетов-прототипов. 

Полученные результаты могут быть использованы для их сравнения с результатами расчетов других самолетов, также в качестве примера расчета параметров системы стабилизации вертикальной скорости.

Все расчеты проводились при помощи MATLAB, Microsoft Excel, Python.
Работа с моделью САУ и её анализ проводились при помощи «Simulink»
