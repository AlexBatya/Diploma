
\subsection{Расчет взлетно-посадочных характеристик самолета}
\label{sec:Расчет взлетно-посадочных характеристик самолета}
\subsubsection{Расчетные формулы и соотношения}

Расчет взлетно-посадочных характеристик производится при следующих
предположениях: 
$\alpha_\text{р} = \alpha_\text{п} = 2^0$ -- угол атаки при разбеге и пробеге \\
$\alpha_\text{отр} = \alpha_\text{кас} = 6^0$ -- угол атаки при отрыве (во время взлёта) и качания ВПП (при пасадке).\\
$H_\text{взл} = 10,7$м, $H_\text{пос} = 15$м -- безопасная высота пролёта препятствий \\
$f_\text{р}$ -- коэффициент трения при разбеге \\
$f_\text{пр} = 0,2$ -- коэффициент трения при пробеге (с учётом тормозной системы), \\
тяга двигателя на взлётном режиме на 15...20\% больше наминальной тяги, при пробеге по ВПП на неманевренном самолёте используется режим "реверса тяги". \\

Скорость отрыва на взлёте, м/c -- 
\begin{equation}
    \label{eq:Скорость отрыва}
    V_\text{отр} = \sqrt{\frac{20P_s(1-0,9\bar{P}_\text{взл}\sin{\alpha_\text{отр}})}{\rho_0C_{y_\text{отр}}}}
\end{equation}

Длина разбега, м -- 
\begin{equation}
    \label{eq:Длина разбега}
    L_\text{р} = \frac{1}{2gb_\text{р}}\ln{\frac{C_\text{p}}{C_\text{р}-b_\text{р}V^2_\text{отр}}}
\end{equation}

Коэффициент тяги -- 
\begin{equation}
    \label{eq:Коэффициент тяги}
    C_\text{р} = 0,9 \bat{P}_\text{взл}-f_\text{р}
\end{equation}

Коэффициент при расчёте длины разега -- 
\begin{equation}
    \label{eq:Коэффициент при расчёте длины разега}
    b_\text{р} = (C_{x_\text{р}}-f_\text{р}C_{y_\text{р}})\frac{\rho_0}{2P_s10}
\end{equation}

Взлётная дистанция, м-- 
\begin{equation}
    \label{eq:Взлётная дистанция}
    L_\text{вд} = L_\text{р} + L_\text{вув}
\end{equation}

Длина воздушного участка взлёта, м -- 
\begin{equation}
    \label{eq:Длина воздушного участка взлёта}
    L_\text{вув} = \frac{1}{\hat{n}_{x_\text{ср}}}(\frac{V_2^2-V^2_\text{отр}}{2g}+H_\text{взл})
\end{equation}

Среднеквадратичное значение скорости, м/с -- 
\begin{equation}
    \label{eq:Среднеквадратичное значение скорости}
    \hat{V}_\text{ср} = \sqrt{\frac{V_2^2-V^2_\text{отр}}{2}}
\end{equation}

Скорость качания ВПП, м/c -- 
\begin{equation}
    \label{eq:Скорость качания ВПП}
    V_\text{кас} = \sqrt{\frac{2\bar{m}_\text{пос}P_s10}{C_{y_\text{кас}}\rho_0}}
\end{equation}

Относительная масса -- 
\begin{equation}
    \label{eq:Относительная масса}
    \bar{m}_\text{пос} = \mkkr-\mtcnp
\end{equation}

Длина пробега, м -- 
\begin{equation}
    \label{eq:Длина пробега}
    L_\text{проб} = \frac{1}{2gb_\text{п}}\ln{\frac{a_\text{п}-b_\text{п}V^2_\text{кас}}{a_\text{п}}}
\end{equation}

Коэффициенты пробега -- 
\begin{equation}
    \label{eq:Коэффициент пробега1}
    a_\text{п} = -\bar{P}_\text{рев} - f_\text{п}
\end{equation}

\begin{equation}
    \label{eq:Коэффициент пробега2}
    b_\text{п} = \frac{\rho_0}{\bar{m}_\text{пос}P_s20}(C_{x_\text{проб}}-f_\text{п}C_{y_\text{проб}})
\end{equation}

Относительный реверс тяги на посадке -- 
\begin{equation}
    \label{eq:Относительный реверс тяги на посадке}
    \bar{P}_\text{рев} = \frac{P_\text{рев}}{m_\text{пос}g}
\end{equation}

Посадочная дистанция, м -- 
\begin{equation}
    \label{eq:Посадочная дистанция}
    L_\text{пд} = L_\text{проб} + L_\text{вуп}
\end{equation}

Длина воздушного участка посадки, м -- 
\begin{equation}
    \label{eq:Длина воздушного участка посадки}
        L_\text{вуп} = K_\text{пос}(\frac{V_\text{пл}^2-V^2_\text{кас}}{2g}+H_\text{пос})
\end{equation}
 
 Скорость планирования, м/c -- 
 \begin{equation}
     \label{eq:Скорость планирования}
     V_\text{пл} = \sqrt{\frac{2\bar{m}_\text{пос}P_s10}{C_{y_\text{пос}}\rho_0}}
 \end{equation}
 
 Аэродинамическое качество при посадке -- 
 \begin{equation}
     \label{eq:Аэродинамическое качество при посадке}
     K_\text{пос} = \frac{C_{y_\text{пос}}}{C_{x_\text{пос}}}
 \end{equation}
 
 \subsection{Результаты расчёта}
 
 \begin{table}[H]
     \centering
     \caption{Результаты расчета взлетно-посадочных характеристик самолета Concorde}
     \begin{tabular}{|c|c|c|c|c|c|}
     \hline
          $V_\text{отр}, $ м/c& $L_\text{р},$ м & $L_\text{вд},$ м& $V_\text{кас},$ м/с  &$L_\text{проб},$ м  &$L_\text{пд},$ м\\ \hline
          88,85& 1125,37 & 1392 & 64,58 & 576 &1200,78\\ \hline
     \end{tabular}
     \label{tab:my_label}
 \end{table}


  
