\subsection{Расчёт характеристик продольной устойчивости самолёта}
\label{sec:Расчёт характеристик продольной устойчивости самолёта}

Прежде чем приступить к расчету характеристик продольной статической устойчивости и управляемости необходимо определить безразмерную площадь горизонтальногооперения $\bar{S}_\text{ГО}$ из условия устойчивости и балансировки.

Далее вычисляется следующие характеристики:
\begin{enumerate}
    \item Зависимости от числа М (скорости) полета:
    \begin{itemize}
        \item [-] запаса статической устойчивости по перегрузке $\sigma_n$,
        \item [-] фокуса самолета $\bar{X}_F$,
        \item [-] точки нейтральности по перегрузке $\bar{X}_H$,
        \item [-] предельно задней центровки $\bar{X}_\text{ТПЗ}$.
    \end{itemize}
    \item Зависимость от числа М (скорости) полета:
    \begin{itemize}
        \item [-] балансировочного отклонения органа продольного управления $\varphi_\text{бал}(\delta_\text{бал})$,
        \item [-] градиента отклонения органа продольного управления по перегрузки
        \item [-] располагаемой нормальной перегрузки из условия балансировки.
    \end{itemize}
\end{enumerate}

Для определения площади ГО ($S_\text{ГО}$) рассчитываются предельно передняя $\bar{X}_\text{ТПП}$ и предельного значения $\bar{X}_\text{ТПЗ}$ центровки.Расчёт предельно передней центровки ведётся для режима пасадки ($H = 0$ км, $M = 0,2$). Предельно передняя центровка вычисляется для указанного режима вычисляется по формуле

\begin{equation}
    \label{eq:Предельно передняя центровка}
    \bar{X}_\text{ТПП} = \frac{-m_{z_0 \text{БГО}} + \bar{X}_{F \ \text{БГО}}  \cdot  C_{y \ \text{БГО}} + C_{y \ \text{ГО}} \cdot \bar{S}_\text{ГО} \cdot K_\text{ГО} \cdot \bar{L}_\text{ГО}}{C_{y \ \text{БГО}}}
\end{equation}

\begin{equation}
    \label{eq:CyБГО}
    C_{y \ \text{БГО}} = C_{y_0 \ \text{БГО}} + C_{y \text{БГО}}^\alpha \cdot \alpha
\end{equation}  

\begin{equation}
    \label{eq:CyГО}
    C_{y \ \text{ГО}} = C_{y \ \text{ГО}}^{\alpha_\text{ГО}}[\alpha(1 - \varepsilon^\alpha) + \varphi_\text{ЭФ}] < 0
\end{equation}

\begin{equation}
    \label{eq:fiЭФ}
    \varphi_\text{ЭФ} = \varphi_\text{уст} + n_B \cdot \delta_{max}
\end{equation}

Аэродинамические характеристики самолета без ГО и ВО для режима посадки указаны в исходных данных.

Величина угла атаки при посадке (касании ВПП) приведена в разделе \ref{sec:Расчет взлетно-посадочных характеристик самолета}. Предельно задняя центровка вычисляется для режима $H=0$ км, $M=0,3$ по формуле

\begin{equation}
    \label{eq:Предельно задняя центровка}
    \bar{X}_\text{ТПЗ} = \bar{X}_\text{Н} + \sigma_{n \ min}
\end{equation}

\begin{equation}
    \label{eq:Нейтральная центровка}
    \bar{X}_H = \bar{X}_F - \frac{m_z^{\bar{\omega}_z}}{\mu}
\end{equation}

\begin{equation}
    \label{eq:Относительная плотность самолёта}
    \mu = 2\frac{P_s}{\rho g b_a}
\end{equation}

\begin{equation}
    \label{eq:Относительный коэффициент демпфирующего момента}
    m_z^{\bar{\omega}_z} = m_z^{\bar{\omega}_{z \ \text{БГО}}} + m_z^{\bar{\omega}_{z \ \text{ГО}}}
\end{equation}

\begin{equation}
    \label{eq:bar(wz)ГО}
    m_z^{\bar{\omega}_z}_{\text{ГО}} = -C_{y \ \text{ГО}}^{\alpha_\text{ГО}} \bar{S}_\text{ГО} \bar{L}_\text{ГО} \sqrt{K_\text{ГО}}
\end{equation}

\begin{equation}
    \label{eq:Относительная координата фокуса}
    \bar{X}_F = \bar{X}_{F \ \text{БГО}} + \Delta \bar{X}_F
\end{equation}

\begin{equation}
    \label{eq:Изменение относительной координаты фокуса}
    \Delta \bar{X}_F \approx \frac{C_{y \ \text{ГО}}^{\alpha_\text{ГО}}}{C_y^\alpha} (1 - \varepsilon^\alpha) \bar{S}_\text{ГО} \bar{L}_\text{ГО} K_\text{ГО}
\end{equation}

$\sigma_{n_{min}} = 0,1 $,  для неманёвренных самолётов

По приведенным выше формулам \ref{eq:Предельно передняя центровка} - \ref{eq:Изменение относительной координаты фокуса} для ряда значений $S_\text{ГО}$ (0.05,0.2) были рассчитаны предельно передняя и предельно задняя центровки.

\begin{center}
    Расчёт $\bar{S}_\text{ГО}$:
\end{center}

\begin{table}[H]
    \centering
        \caption{Предельная задняя и предельная передняя центровки }
    \begin{tabular}{|c|c|c|}
        \hline
        $\bar{S}_\text{ГО}$ & 0,05 & 0,2\\ \hline
        $\bar{X}_\text{ТПП}$ &  & \\ \hline
        $\bar{X}_\text{ТПЗ}$ &  & \\ \hline
    \end{tabular}
    \label{tab:Предельная задняя и предельная передняя центровки}
\end{table}

Затем на одном рисунке строятся зависимости $\bar{X}_\text{ТПП}(\bar{S}_\text{ГО})$, $\bar{X}_\text{ТПЗ}(\bar{S}_\text{ГО})$ графически определяется потребная площадь ГО(см. рис.\ref{fig:Нахождение S_ГО}) из условия: $\bar{X}_\text{ТПЗ}(\bar{S}_\text{ГО}) -\bar{X}_\text{ТПП}(\bar{S}_\text{ГО}) = \Delta \bar{X}_\text{э} \cdot 1,2$, где $\Delta \bar{X}_\text{э}$ -- эксплуатационный разброс центровок 
$\Delta \bar{X}_\text{э} \approx 0,15$ -- для неманёвренных самолётов.

\begin{figure}[H]
    \center{\includegraphics[width=\linewidth]{.jpg}}
    \caption{Нахождение $\bar{X}^*_\text{ГО}$}
    \label{fig:Нахождение S_ГО}
\end{figure}

Далее расчеты характеристик устойчивости и управляемости производятся для
средней центровки: 

\begin{equation}
    \label{eq:Средняя центровка}
    \bar{X}_T = \frac{1}{2} [\bar{X}_\text{ТПЗ}(\bar{S}^*_\text{ГО}) +\bar{X}_\text{ТПП}(\bar{S}^*_\text{ГО})]
\end{equation}
$\bar{S}_\text{ГО}^* = $

Используя ранее приведённые уравнения получим:

$\bar{X}_T = $

При расчете зависимостей $\bar{X}_T(M)$, $\bar{X}_H(M)$, $\bar{X}_\text{ТПЗ}$ используются
формулы \ref{eq:Предельно передняя центровка} - \ref{eq:Изменение относительной координаты фокуса} 

Величина $\sigma_n$ определяется выражением 
\begin{equation}
    \label{eq:Коэффициент продольной статической устойчивости по перегрузке}
    \sigma_n = \bar{X}_T - \bar{X}_F + \frac{m_z^{\bar{\omega}_z}}{\mu}
\end{equation}

Значения величин $\bar{X}_T$, $\bar{X}_F$, $\bar{X}_\text{ТПЗ}$,  $\sigma_n$ определяются в узловых точках $M$ на высоте $H = 0$ км. Результаты данного расчёта оформлены в виде таблицы \ref{tab:Результаты расчётов балансировки в продольном канале}  

\begin{figure}[H]
    \center{\includegraphics[width=\linewidth]{.jpg}}
    \caption{Представление результатов расчета в виде графиков}
    \label{fig:Результаты расчётов балансировки в продольном канале}
\end{figure}

Зависимости $\varphi_\text{бал}(M)[\delta_\text{бал}(V)]$, $\varphi^n(M)[\delta^n(V)]$, $n_{y_P}(M)[n_{y_P}(V)]$ определяютс для трёх значений: $H = 0$ км, $H = 6$ км и высоты крейсерского полёта $H_\text{кр}$, найденных в разделе \ref{sec:Расчет летно-технических характеристик самолета} 
\begin{center}
    Вычисление балансировочного значения руля высоты:
\end{center}

\begin{equation}
    \label{eq:Балансировочное значение руля высоты}
    \delta_\text{бал} = -\frac{m_{x_0}+m^{C_{y \ \text{ГП}}}_{z_{C_{y \ \text{ГП}}}}}{m_z^\delta(1 + \frac{m_z^{C_y}}{\bar{L}_\text{ГО}})}-\frac{\varphi_\text{уст}}{n_\text{В}}
\end{equation}

$$\bar{m} = 1 - \bar{m}_T - \text{относительная масса самолёта}$$

\begin{equation}
    \label{eq:ГП балансировка}
    C_{y_\text{ГП}} = \frac{P_s \cdot \bar{m}}{q}
\end{equation}

\begin{equation}
    \label{eq:mzCy}
    m_z^{C_y} = \bar{X}_T - \bar{X}_F
\end{equation}

\begin{equation}
    \label{eq:Расход руля высоты}
    n_B = \frac{C_y^\delta}{C_y^\varphi}
\end{equation}

\begin{equation}
    \label{eq:Балансировочное значение коэффициента нулевого продольного момента}
    m_{z_0} = m_{z_{0 \ \text{БГО}}} - \bar{S}_\text{ГО} \bar{L}_\text{ГО} K_\text{ГО} C^{\alpha_\text{ГО}}_{y_\text{ГО}} \alpha_0(1 - \varepsilon^\alpha)
\end{equation}

\begin{center}
    Вычисление отклонения стабилизатора на единицу перегрузки:
\end{center}

\begin{equation}
    \label{eq:delta_n}
    \delta^n = -\frac{C_{y_\text{ГП}} \cdot \sigma_n}{m_z^\delta}
\end{equation}

\begin{equation}
    \label{eq:mzdelta}
    m_z^\delta = -C^{\alpha_\text{ГП}}_{y_\text{ГП}} \bat{S}_\text{ГП} \bar{L}_\text{ГП} K_\text{ГП} n_B
\end{equation}

\begin{center}
    Вычисление отклонения стабилизатора на единицу перегрузки:
\end{center}

\begin{equation}
    \label{eq:Располагаемая перегрузка}
    n_{y_\text{Р}} = 1 + \frac{(\delta_{max}) - (\delta_\text{бал})}{\delta^n}
\end{equation}

Результаты расчетов оформлены в виде таблиц \ref{tab:} - \ref{tab:}



