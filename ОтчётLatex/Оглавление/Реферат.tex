
\newpage
\begin{center}
    \section*{Реферат}
\end{center}
\pagestyle{fancy}
\fancyhf{}
\rfoot{\thepage}
Дипломная работа «Улучшение рабастности динамической системы в продольном канале управления с применением обратной динамики» \pageref{LastPage} с.

РАСЧЕТ ЛТХ, РАСЧЕТ ВПХ, СИНТЕЗ СИСТЕМЫ АВТОМАТИЧЕСКОГО
УПРАВЛЕНИЯ, ОБРАТНАЯ ДИНАМИКА

Объект исследования -- самолет -- прототип Concorde.

Цель работы -- рассмотреть один из основных способов улучшения робастности динамической системы c применением обратной динамики при помощи PI-котроллера. Сначала в данной работе производится расчет ЛТХ, ВПХ, расчет
характеристик маневренности, после чего производится синтез системы
автоматического управления для самолета-прототипа Concorde, далее - 
рассмотрение непосредственно темы данной дипломной работы.

Данные для расчетов были взяты из \cite{Album} для различных самолетов-прототипов.
 результаты могут быть использованы для их сравнения с
результатами расчетов других самолетов, также в качестве примера расчета
ЛТХ, параметров САУ.

Все расчеты проводились при помощи MATLAB, Python. 

Работа с моделью САУ и её анализ проводились при помощи «Simulink». 

Отчёт оформлен с применением языка разметки Latex.
 