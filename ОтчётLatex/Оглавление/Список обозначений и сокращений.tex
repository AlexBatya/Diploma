\newpage
\pagestyle{fancy}
\fancyhf{}
\rhead{Дипломная работа}
\lhead{Список обозначений}
    \rfoot{ \thepage}
\begin{center}
     \section*{Сокращения и обозначения}
\end{center}
%Сюда я понапихаю все сокращениия по формулам
%обозначения
\newcommand{\Mza}{\bar{M}_z^\alpha}
\newcommand{\Mzwz}{\bar{M}_z^{\omega_z}}
\newcommand{\mza}{m_z^\alpha}
\newcommand{\mzwz}{m_z^{\omega_z}}
\newcommand{\mzf}{m_z^{\delta_{\text{э}}}}
\newcommand{\Cya}{C_y^\alpha}
\newcommand{\Mzf}{\bar{M}_z^{\delta_\text{э}}}
\newcommand{\mzwzch}{m_z^{\bar{\omega}_z}}
\newcommand{\mkkr}{\bar{m}_{\text{к}_\text{кр}}}
\newcommand{\mtcnp}{\bar{m}_{\text{Т}_\text{снп}}}
%единицы измерения
\newcommand{\ones}{\frac{1}{с}}
$a_H$ -- скорость звука на высоте полёта, м/с; \\
$n_{y_\text{э}}$ -- эксплуатационная перегрузка; \\
$V_2$ -- безопасная скорость взлёта, м/с;\\
$V_i$ -- индикаторная скорость самолёта, м/c; \\ 
$H_\text{взл}$ -- безопасная высота взлёта, м;\\
$\hat{n}_{x_\text{ср}}$ -- тангенциальная перегрузка для среднеквадратичного значения скорости;\\
$L_\text{вув}$ -- длина воздушного участка взлёта, м;\\
$\mkkr$ -- относительная масса самолёта в конце крейсерского полёта;\\
$\mtcnp$ -- относительная масса топлива расходуемого при снижении и посадке; \\
$\Cya$ -- производная коэффициента подъёмной силы по углу атаки, 1/рад; \\ 
$\bar{Y}^{\alpha}$ -- относительная подъёмная сила по углу атаки, c; \\
$g$ -- ускорение свободного падения, м/с$^2$; \\ 
$H$ -- высота полёта, км; \\ 
$h_0$ -- коэффициент демпфирования в продольном канале, 1/с;\\
$J_z$ -- момент инерции самолёта относительно оси OZ, кг $\cdot$ м$^2$; \\ 
$K_{\omega_z}$ -- коэффициент демпфирования угловой скорости тангажа, с;  \\
$K_{\vartheta}$ -- коэффициент стабилизации угла тангажа;\\
$K_H$ -- коэффициент усиления ПФ по высоте, м/с;\\
$i_H$ -- коэффициент усиления в статическом законе управления 1/м;\\
$i_p$ -- коэффициент усиления в астатическом законе управления с/м;\\
$b_a$ -- длина САХ крыла, м; \\ 
$M$ -- число Маха;\\
$M_{min \ P}$ -- число М при минимальной тяге, на которой возможен ГП; \\
$M_{max \ P}$ -- число М при максимальной тяге, на которой возможен ГП;\\
$\Mza$ -- производная относительного момента тангажа по углу атаки, 1/(рад $\cdot$ с$^2$);\\
$\Mzwz$ -- относительный демпфирующий момент тангажа, 1/(рад $\cdot$ с);\\
$\Mzf$ -- производная относитльного момента тангажа по углу отклонения флаперонов,\\ 1/(рад $\cdot$ c$^2$); \\ 
$\mza$ -- производная коэффициента момента тангажа по углу атаки; 1/рад; \\  
$\mzwz$ -- производная коэффициента момента тангажа по угловой скорости тангажа; \\
$\mzwzch$ -- производная коэффициента момента тангажа по безразмерной угловой скорости тангажа;\\
$\mzf$ -- производная коэффициента момента тангдажа по отклонению флаперонов, 1/рад;\\
$p_H$ -- давление на высоте H, Па;\\
$P_s$ -- нагрузка на крыло, Н/м;\\
$q$ -- скоростной напор, Н/м$^2$; \\ 
$S$ -- площадь крыла, м$^2$; \\
$\bar{P_0}$ -- стартовая тяговооруженность; \\ 
$P_0$ -- стартовая тяга двигателя, Н; \\
$P$ -- тяга двигателя, Н; \\ 
$T_\text{п}$ -- постоянная времени привода, с; \\ 
$V$ -- воздушная скорость, м/с; \\ 
$m$ -- расчётная масса самолёта; \\
$A$ -- матрица состояний; \\ 
$B$ -- матрица вхдных параметров; \\ 
$C$ -- матрица наблюдений; \\ 
$D$ -- матрица смещений; \\  
$c$ -- функция лётчика; \\
$e$ -- функция ошибки; \\ 
$n_e$ -- ремнанта от лётчика; \\  
$i$ -- функция фходного сигнала; \\ 
$y$ -- функция идущая от объекта управления; \\ 
$\sigma_e^2$ -- квадрат среднего значения разброса случайной величины; \\ 
$\mu_e$ -- математическое ожидание; \\ 
$W_\text{л} - $  передаточная функция лётчика; \\ 
$W_\text{с} - $  передаточная функция объекта управления; \\ 
$S_{n_e  n_e}$ -- спектральная плотность ремнанты; \\ 
$S_{ii}$ -- спектральная плотность входного сигнала; \\ 
$\alpha$ -- угол атаки, град; \\ 
$\vartheta$ -- угол тангажа, град; \\ 
$\delta_\text{э}$ -- угол отклонения элевонов, град; \\ 
$\alpha_0$ --угол атаки при $C_y = 0$, град;\\
$\beta$ -- угол скольжения, град; \\
$\gamma$ -- угол крена, град;\\
$\psi$ -- угол рысканья, град;\\ 
$\delta_\text{э}$ -- угол отклонения элевонов, град;\\
$\rho_H$ -- плотность воздуха на высоте $H$, кг/м$^3$;\\
$n_y^\alpha$ – производная нормальной перегрузки по углу атаки, 1/рад \\
$\omega_0$ -- собственная частота колебаний продольного движения, 1/с; \\ 
$\omega_\text{п}$ -- собственная частота привода, 1/с; \\ 
$\omega_z$ -- угловая скорость тангажа, рад/с; \\ 
$\omega_\text{п}$ -- собственная частота привода, 1/c; \\
$\xi_\text{п}$ -- коэффициент относительного демпфирования привода; \\
ЛА -- летательный аппарат; \\ 
ПФ -- передаточная функция; \\ 
ГП -- горизонтальный полёт; \\
