\begin{frame}{Расчёт ЛТХ}
\begin{block}{В расчёт ЛТХ входит}
   \begin{enumerate}
    \item Расчёт области установившихся горизонтальных полётов  
    \item Расчёт траектории полёта 
    \item Расчёт транспортных возможностей самолёта
   \end{enumerate}
\end{block}
\end{frame}

% \begin{frame}{Расчёт ЛТХ}{Расчёт области возможных полётов}

%     \begin{block}{Основные ограничения}
%         \begin{itemize}
%             \item Ограничение по $M_{min \ P}$ 
%             \item Ограничение по $M_{max \ P}$
%             \item Ограничение по $C_{y \ \text{доп}}$
%             \item Ограничение по $M_\text{пред}$
%             \item Ограничение по $q_{max}$
%         \end{itemize}
%     \end{block}

% \end{frame}

% \begin{frame}{Расчёт ЛТХ}{Результаты расчётов $M_{C_{y \ \text{доп}}}$ и $M_{min \ P}$, $M_{max \ P}$, $M_\text{наев}$}
%     \begin{minipage}[c]{0.45\textwidth}
%         \center{\includegraphics[width=6cm, height = 7cm]{../Оглавление/Part1/figures/CyCydop2.jpg}}
%     \end{minipage}
%     \begin{minipage}[c]{0.45\textwidth}
%         \center{\includegraphics[width=6cm, height = 7cm]{../Оглавление/Part1/figures/PpPr2.jpg}}
%     \end{minipage}
% \end{frame}



\begin{frame}{Расчёт ЛТХ}{Расчёт области возможных полётов}
        \center{\includegraphics[width=11cm, height = 7cm]{../Оглавление/Part1/figures/Область.jpg}}
\end{frame}

\begin{frame}{Расчёт ЛТХ}{Определение теоретического и практического потолка}
    \begin{minipage}[c]{0.45\textwidth}
        \begin{block}{Потолки}
        \begin{itemize}
            \item  [] Расчёт статического и практического потолка производится по $V^*_{y_{max}}$
            \item  [] $H_\text{т} = 19,8$ км 
            \item  [] $H_\text{пр} = 19,5$ км
        \end{itemize}
        \end{block}
    \end{minipage}
    \begin{minipage}[c]{0.45\textwidth}
        \center{\includegraphics[width=6cm, height = 7cm]{../Оглавление/Part1/figures/Vy(H).jpg}}
    \end{minipage}
\end{frame}

\begin{frame}{Расчёт ЛТХ}{Минимальные значения часового и километрового расходов топлива}
        \center{\includegraphics[width=10cm, height = 7cm]{../Оглавление/Part1/figures/qhqkm_MAX.jpg}}
\end{frame}

\begin{frame}{Расчёт ЛТХ}{Расчёт траектории полёта}
    \begin{block}{Траектория}
    \begin{itemize}
        \item [] Траеткорию полёта принято разделять на три этапа 
            \begin{itemize}
                \item Набор высоты 
                \item Крейсерский полёт 
                \item Снижение 
            \end{itemize}
    \end{itemize}
    \end{block}
\end{frame}

% \begin{frame}{Расчёт ЛТХ}{Расчёт траектории набора}
%     \begin{block}{Выбор начальных параметров}
%         Начальные значения H и М определяются следующим образом: $H_0$ = 0 км $M_0$ = 1,2$\cdot M_{min \ \text{доп}}$, а конечные значения выбираются из
%         условия минимума километрового расхода топлива в установившемся горизонтальном полете. Высота и число Маха, при которых километровый расход 
%         топлива принимает наименьшее значение, определены в предыдущих слайдах 
%     \end{block}
% \end{frame}

% \begin{frame}{Расчёт ЛТХ}{Расчёт траектории набор}
%     \begin{minipage}[c]{0.45\textwidth}
%         \center{\includegraphics[width=6cm, height = 7cm]{../Оглавление/Part1/figures/Характеристики набора высоты1.jpg}}
%     \end{minipage}
%     \begin{minipage}[c]{0.45\textwidth}
%         \center{\includegraphics[width=6cm, height = 7cm]{../Оглавление/Part1/figures/Характеристики набора высоты2.jpg}}
%     \end{minipage}
% \end{frame}
 
% \begin{frame}{Расчёт ЛТХ}{Расчёт траектории набора}
%     \begin{block}{Результаты расчётов}
%     \begin{table}
%         \begin{tabular}{|c|c|c|}
%             \hline
%             Параметр&  Значение & Единицы  \\ \hline
%             $m_{T_{\text{наб}}}$& 7225,6 &кг \\ \hline
%             $L_\text{наб}$& 278,04 &км \\ \hline
%             $T_\text{наб}$& 20,06 & мин\\ \hline
%         \end{tabular}
%     \end{table}
%     \end{block}
% \end{frame}

% \begin{frame}{Расчёт ЛТХ}{Расчёт крейсерского полёта}
%     \begin{block}{Выбор начальных параметров}
%         $\bar{m}_{T_\text{наб}}$ = 0,5 – относительная масса пустого снаряженного самолета \\ 
%         $\bar{m}_\text{цн}$ = 0,15 – относительная масса целевой нагрузки \\ 
%         $\bar{m}_\text{снп}$ = 0,015 – относительная масса топлива 
%         расходуемая при снижении и посадке  \\
%         $\bar{m}_{T_\text{наб}}$– относительная масса топлива, расходуемая при наборе высоты \\
%     \end{block}
%     \begin{itemize}
%     \item [] \begin{block}{Результаты расчётов характеристик крейсерского полёта}
%         \begin{table}
%             \begin{tabular}{|c|c|c|}
%                 \hline
%                 Параметр& Значение & Единицы\\ \hline
%                 $H_\text{к \ кр}$& 19.3 & км\\ \hline
%                 $L_\text{кр}$& 7610,74 &кг \\ \hline
%                 $T_\text{кр}$& 403,43 & мин\\ \hline
%             \end{tabular}
%         \end{table}
%     \end{block}
%     \end{itemize}
% \end{frame} 

% \begin{frame}{Расчёт ЛТХ}{Расчёт траектории спуска} %Предупреждение
%         \begin{minipage}[c]{0.45\textwidth}
%             \center{\includegraphics[width=6cm, height = 7cm]{../Оглавление/Part1/figures/Характеристики спуска1.jpg}}
%         \end{minipage}
%         \begin{minipage}[c]{0.45\textwidth}
%             \center{\includegraphics[width=6cm, height = 7cm]{../Оглавление/Part1/figures/Характеристики пуска2.jpg}}
%         \end{minipage}
% \end{frame}

% \begin{frame}{Расчёт ЛТХ}{Расчёт траектории спуска}
%     \begin{block}{Результаты расчётов}
%         \begin{table}
%             \begin{tabular}{|c|c|c|}
%                 \hline
%                 Параметр& Значение & Единицы\\ \hline
%                 $m_{T_\text{спуск}}$&756,936 & кг\\ \hline
%                 $L_\text{спуск}$& 314,16 &км \\ \hline
%                 $T_\text{спусе}$& 41,929 & мин\\ \hline
%             \end{tabular}
%         \end{table}
%     \end{block}
% \end{frame}
\begin{frame}{Расчёт ЛТХ}{Расчёт траектории полёта}
    \begin{block}{Результаты расчётов}
    $$L = L_\text{наб} + L_\text{кр} + L_\text{спуск} = 278,04 \text{ км} + 7610 \text{ км} + 314,16 \text{ км} = 8202,2 \text{ км}$$ 
        $$T = T_\text{наб} + T_\text{кр} + T_\text{спуск} = 20,06 \text{ мин} + 403 \text{ мин} + 42 \text{ мин} = 465,4 \text{ мин}$$ 
        $$m_T = m_{T_\text{наб}} + m_{T_\text{кр}} + m_{T_\text{спуск}} = 7225 \text{ кг} + 50 234 \text{ кг} + 757 \text{ кг}= 58216 \text{ кг}$$
    \end{block}
\end{frame}

\begin{frame}{Расчёт ЛТХ}{Расчёт траектории полёта}
        \center{\includegraphics[width=10cm, height = 7cm]{../Оглавление/Part1/figures/Траектория полета.jpg}}
\end{frame}


\begin{frame}{Расчёт ЛТХ}{Расчёт транспортных возможностей самолёта}
    \begin{itemize}
        \item  [] \begin{block}{Основные положения}
            \begin{itemize}
            \item  [] Расчёт ведётся для трёх режимов
            \begin{itemize}
                \item  Полет с максимальной коммерческой нагрузкой
                \item Полёт с максимальным запасом топлива
                \item Полёт без коммерческой нагрузки ($m_\text{цн}$ = 0) с максимальным запасом топлива
            \end{itemize}
        \end{itemize}
        \end{block}
    \end{itemize}
\end{frame}

\begin{frame}{Расчёт ЛТХ}{Диаграмма транспортных возможностей самолёта}
    \center{\includegraphics[width=10cm, height = 7cm]{../Оглавление/Part1/figures/Транспортные возможности.jpg}}
\end{frame}