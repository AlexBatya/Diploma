\documentclass[unicode]{beamer}   %Это нам необходимо для объявления призентации
\usepackage{cmap}
\usepackage[utf8]{inputenc}
\usepackage[T2A]{fontenc}
\usepackage[english,russian]{babel} %локализация и переносы  
\usepackage{amsthm,amsmath,amssymb}
\usetheme{Madrid}
% \setromanfont{Times New Roman}
\setbeamercolor{titlelike}{parent=structure,bg=gray,fg = white}
\definecolor{myNewColorA}{RGB}{160,160,160}
\definecolor{myNewColorB}{RGB}{255,255,255}
\definecolor{myNewColorC}{RGB}{160,160,160}
\definecolor{myNewColorD}{RGB}{160,160,160}

\setbeamercolor*{palette primary}{bg=myNewColorA, fg = white}
\setbeamercolor*{palette secondary}{bg=myNewColorB, fg = gray}
\setbeamercolor*{palette tertiary}{bg=myNewColorC, fg = white}
\setbeamercolor*{palette quaternary}{bg=myNewColorD, fg = green}


\title[Дипломная работа]{Улучшение робастности динамической системы в продольном канале управления с применением обратной динамики}
\subtitle[Краткое название]{Полное название}
\author{Выполнил: Пащенко А.Е.}
\date{}
\logo{\includegraphics[width = 1cm]{img/Герб.png}}

\begin{document}
\maketitle


    % НАЧАЛО ТИТУЛЬНОГО ЛИСТА
\begin{center}
    \hfill \break
    \large{МИНИСТЕРСТВО НАУКИ И ВЫСШЕГО ОБРАЗОВАНИЯ РОССИЙСКОЙ ФЕДЕРАЦИИ }\\
    \footnotesize{ФЕДЕРАЛЬНОЕ ГОСУДАРСТВЕННОЕ БЮДЖЕТНОЕ ОБРАЗОВАТЕЛЬНОЕ УЧРЕЖДЕНИЕ}\\ 
    \footnotesize{ВЫСШЕГО ПРОФЕССИОНАЛЬНОГО ОБРАЗОВАНИЯ}\\
    \small{\textbf{«МОСКОВСКИЙ АВИАЦИОННЫЙ ИНСТИТУТ»\\(Национальный исследовательский университет)}}\\ \hline
    \hfill \break
    \normalsize{Институт №1}\\
    \normalsize{Кафедра 106}\\
    \hfill\break

    \large{\textbf{КУРСОВАЯ РАБОТА} \\ по дисциплине «Управление движением ЛА» }\\

    \normalsize{Тема: «Система стабилизации вертикальной скорости самолета»}\\
\end{center}
\hfill \break

\normalsize{ 
\begin{flushleft}
    \underline{Проверил:} \\
    \hfill \break  
    Ассистент каф. 106 Е.В. Ефремов \\ \hfill \break
    «\underline{\hspace{1cm}}» \underline{\hspace{3cm}} 2022 г. \\ \hfill \break
    \underline{Выполнил:} \\ \hfill \break 
    Студент каф. 106 гр. 403Б А.Е. Пащенко \\
    «\underline{\hspace{1cm}}» \underline{\hspace{3cm}} 2022 г. \\ \hfill \break
\end{flushleft}
}\\

\begin{center} Москва 2022 \end{center}
\thispagestyle{empty} % выключаем отображение номера для этой страницы!!
 
% КОНЕЦ ТИТУЛЬНОГО ЛИСТА

\newpage
\begin{center}
    \section*{Реферат}
\end{center}

Дипломная работа «Улучшение рабастности динамической системы в продольном канале управления с применением обратной динамики» \pageref{lastPage}~с.
\begin{frame}{Объект исследования}

    \begin{figure}[H]
        \center{\includegraphics[width=12cm, height = 7cm]{../img/Concorde.jpg}}
        % \caption{Concorde}
        % \label{fig:Concorde}
    \end{figure}

\end{frame}

\begin{frame}{Расчёт ЛТХ}
\begin{block}{В расчёт ЛТХ входит}
   \begin{enumerate}
    \item [] <+->
    \item <+-> Расчёт области возможных полётов
    \item <+-> Расчёт траектории полёта 
    \item <+-> Расчёт транспортных возможностей самолёта
   \end{enumerate}
\end{block}
\end{frame}

\begin{frame}{Расчёт ЛТХ}{Расчёт области возможных полётов}

    \begin{block}{Основные ограничения}
        \begin{itemize}
            \item Ограничение по $M_{min \ P}$ 
            \item Ограничение по $M_{max \ P}$
        \end{itemize}
    \end{block}

    \begin{block}{Дополнительные ограничения}
        \begin{itemize}
            \item Ограничение по $C_{y \ \text{доп}}$
            \item Ограничение по $M_\text{пред}$
            \item Ограничение по $q_{max}$
        \end{itemize}
    \end{block}

\end{frame}

\begin{frame}{Расчёт ЛТХ}{Результаты расчётов $M_{C_{y \ \text{доп}}}$ и $M_{min \ P}$, $M_{max \ P}$, $M_\text{наев}$}
    \begin{minipage}[c]{0.45\textwidth}
        \center{\includegraphics[width=6cm, height = 7cm]{../Оглавление/Part1/figures/CyCydop2.jpg}}
    \end{minipage}
    \begin{minipage}[c]{0.45\textwidth}
        \center{\includegraphics[width=6cm, height = 7cm]{../Оглавление/Part1/figures/PpPr2.jpg}}
    \end{minipage}
\end{frame}

\begin{frame}{Расчёт ЛТХ}{Результаты расчётов $q_{\text{ч} \ min}$ и $q_{\text{км} \ min}$}
    \begin{minipage}[c]{0.45\textwidth}
        \center{\includegraphics[width=6cm, height = 7cm]{../Оглавление/Part1/figures/qh2.jpg}}
    \end{minipage}
    \begin{minipage}[c]{0.45\textwidth}
        \center{\includegraphics[width=6cm, height = 7cm]{../Оглавление/Part1/figures/qkm2.jpg}}
    \end{minipage}
\end{frame}

\begin{frame}{Расчёт ЛТХ}{Результаты расчётов $M_{V_{y \ max}}$}
        \center{\includegraphics[width=10cm, height = 7cm]{../Оглавление/Part1/figures/Vy2.jpg}}
\end{frame}

\begin{frame}{Расчёт ЛТХ}{Расчёт области возможных полётов}
    \begin{minipage}[c]{0.55\textwidth}
        \center{\includegraphics[width=6cm, height = 7cm]{../Оглавление/Part1/figures/Область.jpg}}
    \end{minipage}
    \begin{minipage}[c]{0.4\textwidth}
        \begin{itemize}
            \item <+-> []
            \item <+-> [] \begin{block}{Определение области}
                \begin{itemize}
                    \item $M_{min} = max \{ M_{min \ p}, \ M_{C_{y \ \text{доп}}} \} $
                    \item $M_{max} = min \{ M_{max \ P}, \ M_{\text{пред}}, \ M_{q_{max}} \}$
                \end{itemize}
            \end{block}
        \end{itemize}
    \end{minipage}
\end{frame}

\begin{frame}{Расчёт ЛТХ}{Определение теоретического и практического потолка}
    \begin{minipage}[c]{0.45\textwidth}
        \begin{block}{Потолки}
        \begin{itemize}
            \item <+-> []
            \item <+-> [] Расчёт теоретического и практического потолка производится по $V^*_{y_{max}}$
            \item <+-> [] $H_\text{т} = 19,8$ км 
            \item <+-> [] $H_\text{пр} = 19,5$ км
        \end{itemize}
        \end{block}
    \end{minipage}
    \begin{minipage}[c]{0.45\textwidth}
        \center{\includegraphics[width=6cm, height = 7cm]{../Оглавление/Part1/figures/Vy(H).jpg}}
    \end{minipage}
\end{frame}

\begin{frame}{Расчёт ЛТХ}{Максимальные значения часового и километрового расходов}
        \center{\includegraphics[width=10cm, height = 7cm]{../Оглавление/Part1/figures/qhqkm_MAX.jpg}}
\end{frame}

\begin{frame}{Расчёт ЛТХ}{Расчёт траектории полёта}
    \begin{block}{Траектория}
    \begin{itemize}
        \item [] <+->
        \item [] <+-> Траеткорию полёта принято разделять на три этапа 
            \begin{itemize}
                \item Набор высоты 
                \item Крейсерский полёт 
                \item Снижение 
            \end{itemize}
    \end{itemize}
    \end{block}
\end{frame}

\begin{frame}{Расчёт ЛТХ}{Расчёт траектории набора}
    \begin{block}{Выбор начальных параметров}
        Начальные значения H и М определяются следующим образом: $H_0$ = 0 км $M_0$ = 1,2$\cdot M_{min \ \text{доп}}$, а конечные значения выбираются из
        условия минимума километрового расхода топлива в установившемся горизонтальном полете. Высота и число Маха, при которых километровый расход 
        топлива принимает наименьшее значение, определены в предыдущих слайдах 
    \end{block}
\end{frame}

\begin{frame}{Расчёт ЛТХ}{Расчёт траектории набор}
    \begin{minipage}[c]{0.45\textwidth}
        \center{\includegraphics[width=6cm, height = 7cm]{../Оглавление/Part1/figures/Характеристики набора высоты1.jpg}}
    \end{minipage}
    \begin{minipage}[c]{0.45\textwidth}
        \center{\includegraphics[width=6cm, height = 7cm]{../Оглавление/Part1/figures/Характеристики набора высоты2.jpg}}
    \end{minipage}
\end{frame}
 
\begin{frame}{Расчёт ЛТХ}{Расчёт траектории набора}
    \begin{block}{Результаты расчётов}
    \begin{table}
        \begin{tabular}{|c|c|c|}
            \hline
            Параметр&  Значение & Единицы  \\ \hline
            $m_{T_{\text{наб}}}$& 7225,6 &кг \\ \hline
            $L_\text{наб}$& 278,04 &км \\ \hline
            $T_\text{наб}$& 20,06 & мин\\ \hline
        \end{tabular}
    \end{table}
    \end{block}
\end{frame}

\begin{frame}{Расчёт ЛТХ}{Расчёт крейсерского полёта}
    \begin{block}{Выбор начальных параметров}
        $\bar{m}_{T_\text{наб}}$ = 0,5 – относительная масса пустого снаряженного самолета \\ 
        $\bar{m}_\text{цн}$ = 0,15 – относительная масса целевой нагрузки \\ 
        $\bar{m}_\text{снп}$ = 0,015 – относительная масса топлива 
        расходуемая при снижении и посадке  \\
        $\bar{m}_{T_\text{наб}}$– относительная масса топлива, расходуемая при наборе высоты \\
    \end{block}
    \begin{itemize}
        \item <+-> []
    \item <+-> [] \begin{block}{Результаты расчётов характеристик крейсерского полёта}
        \begin{table}
            \begin{tabular}{|c|c|c|}
                \hline
                Параметр& Значение & Единицы\\ \hline
                $H_\text{к \ кр}$& 19.3 & км\\ \hline
                $L_\text{кр}$& 7610,74 &кг \\ \hline
                $T_\text{кр}$& 403,43 & мин\\ \hline
            \end{tabular}
        \end{table}
    \end{block}
    \end{itemize}
\end{frame} 

\begin{frame}{Расчёт ЛТХ}{Расчёт траектории спуска} %Предупреждение
        \begin{minipage}[c]{0.45\textwidth}
            \center{\includegraphics[width=6cm, height = 7cm]{../Оглавление/Part1/figures/Характеристики спуска1.jpg}}
        \end{minipage}
        \begin{minipage}[c]{0.45\textwidth}
            \center{\includegraphics[width=6cm, height = 7cm]{../Оглавление/Part1/figures/Характеристики пуска2.jpg}}
        \end{minipage}
\end{frame}

\begin{frame}{Расчёт ЛТХ}{Расчёт траектории спуска}
    \begin{block}{Результаты расчётов}
        \begin{table}
            \begin{tabular}{|c|c|c|}
                \hline
                Параметр& Значение & Единицы\\ \hline
                $m_{T_\text{спуск}}$&756,936 & кг\\ \hline
                $L_\text{спуск}$& 314,16 &км \\ \hline
                $T_\text{спусе}$& 41,929 & мин\\ \hline
            \end{tabular}
        \end{table}
    \end{block}
\end{frame}

\begin{frame}{Расчёт ЛТХ}{Расчёт траектории полёта}
    \center{\includegraphics[width=10cm, height = 8cm]{../Оглавление/Part1/figures/Траектория полета.jpg}}
\end{frame}

\begin{frame}{Расчёт ЛТХ}{Расчёт транспортных возможностей самолёта}
    \begin{itemize}
        \item <+-> []
        \item <+-> [] \begin{block}{Основные положения}
            \begin{itemize}
            \item <+-> [] Расчёт ведётся для трёх режимов
            \begin{itemize}
                \item <+-> Полет с максимальной коммерческой нагрузкой
                \item <+-> Полёт с максимальным запасом топлива
                \item <+-> Полёт без коммерческой нагрузки ($m_\text{цн}$ = 0) с максимальным запасом топлива
            \end{itemize}
        \end{itemize}
        \end{block}
    \end{itemize}
\end{frame}

\begin{frame}{Расчёт ЛТХ}{Диаграмма транспортных возможностей самолёта}
    \center{\includegraphics[width=10cm, height = 8cm]{../Оглавление/Part1/figures/Транспортные возможности.jpg}}
\end{frame}
% \begin{frame}{Синтез системы автоматического управления}
    \begin{itemize}
        \item <+-> []
        \item <+-> []  \begin{block}{Задачи раздела}
                Расчет коэффициентов и моделирование системы стабилизации вертикальной скорости самолета для Concorde: 
            \begin{itemize}
                \item Выбор параметров привода 
                \item Расчет и оценка коэффициентов обратных связей и коэффициентов стабилизации системы
                \item Частотный анализ контуров системы 
                \item Моделирование и анализ линейной и нелинейной САУ
            \end{itemize}
        \end{block}
    \end{itemize}
\end{frame}

\begin{frame}{Исследуемая модель} %Какаета ошибка
    $$ \begin{cases}
            \dot{\alpha}=\omega_z-\bar{Y}^{\alpha} \alpha \\
            \dot{\omega}_z=\bar{M}_z^{\alpha} \alpha+\bar{M}_z^{\omega_z} \omega_z +\bar{M}_z^{\dot{\alpha}} \dot{\alpha}+\bar{M}_z^{\delta_{\text{в}}} \delta_{\text{в}} \\
            \dot{V_y}=V \cdot \bar{Y}^{\alpha} \alpha
    \end{cases} $$ \\
    $A = \begin{pmatrix}
        -\bat{Y^{\alpha}} & 1 & 0\\ 
        \bar{M}_z^\alpha & \bar{M}_z^{\omega_z} & 0\\ 
         V \cdot \bar{Y}^\alpha& 0 & 0 
    \end{pmatrix}$;
    $B = \begin{pmatrix}
     0 \\ 
     \bar{M}_z^{\delta_{\text{э}}} \\ 
     0 
    \end{pmatrix}$ ;
    $C= \begin{pmatrix}
    1 & 0 & 0\\ 
    0 & 1 & 0\\ 
     0& 0 &1 
    \end{pmatrix}$;
    $D = \begin{pmatrix}
     0 \\ 
     0 \\ 
     0 
    \end{pmatrix}$
\end{frame}

\begin{frame}{Структурная схема системы стабилизации вертикальной скорости самолета}
    \center{\includegraphics[width = \linewidth]{../Оглавление/Part2/Sactions/Content/figures/Схема.jpg}}
\end{frame}

\begin{frame}{Выбор параметров привода}
    \begin{itemize}
    \item <+-> []
    \item <+-> []   \begin{block}{Передаточная функция привода}
        При решении задачи синтеза сервопривод описывается передаточной функций колебательного звена:
    
    \begin{equation}
    \label{eq:Привод ограничеия}
        W_{\text{п}}=\frac{1}{T_\text{п}^2p^2+2\xi_\text{п}T_\text{п}p+1}
    \end{equation}
    
    Значение постоянной времени  $T_\text{п}$ сервопривода, от которой зависит его полоса пропускания, определяется следующим образом:
    
    Устанавливается максимальное значение собственной частоты  недемпфированных колебаний $\omega_0=\frac{1}{T_{\text{с}}}$ в варианте управлении продольным движением самолета, и исходя из этих значений, определяется потребная ширина полосы пропускания сервопривода (см. формула \ref{eq:Привод ограничеия}):
    \end{block}
\end{itemize}
\end{frame}

\begin{frame}{Выбор параметров привода}
    \begin{block}{Вывод}
        \begin{itemize}
            \item Максимальное значение $\omega_0$ находится у поверхности земли со значением $M = 1$ ($\omega_0_{max} = 5,74 \ \frac{1}{c}$).
            \item $\omega_\text{п} = 37,19 \ \frac{1}{c}$ => $T_{\text{п}} = 0.0269 \ c$ 
            \item Из данного ряда чисел [0,02; 0,025; 0,003; 0,035; 0,04; 0,045; 0,05] 0,0269 более близко к 0,025, следовательно, данное число мы и примем за постоянную времени привода. Исходя из вышесказанного, получаем $\omega_\text{п} = 40 \ \frac{1}{c}$ , $T_{\text{п}} = 0.025 \ c, \xi = 0,5$.
        \end{itemize}
    \end{block}
\end{frame}


\begin{frame}{Расчёт коэффициентов стабилизации системы}
    \begin{minipage}[c]{0.45\textwidth}
        \center{\includegraphics[width=6cm, height = 7cm]{../Оглавление/Part2/Sactions/Content/figures/K_wz.jpg}}
    \end{minipage}
    \begin{minipage}[c]{0.45\textwidth}
        \center{\includegraphics[width=6cm, height = 7cm]{../Оглавление/Part2/Sactions/Content/figures/K_v.jpg}}
    \end{minipage}
\end{frame}

\begin{frame}{Расчёт коэффициентов стабилизации системы}
    \begin{minipage}[c]{0.45\textwidth}
        \center{\includegraphics[width=6cm, height = 7cm]{../Оглавление/Part2/Sactions/Content/figures/i_H.jpg}}
    \end{minipage}
    \begin{minipage}[c]{0.45\textwidth}
        \center{\includegraphics[width=6cm, height = 7cm]{../Оглавление/Part2/Sactions/Content/figures/i_p.jpg}}
    \end{minipage}
\end{frame}

\begin{frame}{Расчёт коэффициентов стабилизации системы}
    \begin{block}{Вывод}
        Полученные значения коэффициентов обратных связей были успешно найдены и применены на модели рассматриваемой 
        системы стабилизации вертикальной скорости в системе «Simulink». Моделирование показало, что коэффициенты найдены верно, 
        так как заданная вертикальная скорость равена вертикальной скорости на выходе из системы. 
        Более подробно будут показаны результаты моделирования и сама модель в разделе «Нелинейное моделирование».
    \end{block}
\end{frame}

\begin{frame}{Моделирование и анализ линейной и нелинейной САУ}
    \begin{itemize}
        \item <+-> []
        \item <+-> []
    \begin{block}{Основные положения}
        Целью частотного анализа является построение логарифмических амплитудных и фазовых частотных характеристик (ЛАФЧХ) 
        разомкнутых и замкнутых контуров управления до синтеза и после синтеза и проведение их сравнительного анализа.
    \end{block}
    \item <+-> []
    \begin{block}{Примечание}
        В данной призентации будет приведены частотные характеристики только для крейсерского полёта, 
        для остальных режимов всё аналогично.
    \end{block}
\end{itemize}
\end{frame}

\begin{frame}{Частотный аналез крейсерского режима полёта}
    \center{\includegraphics[width=10cm, height = 8cm]{../Оглавление/Part2/Sactions/Content/frequencies/Вертикальная скорость раз qKR.jpg}}
\end{frame}

\begin{frame}{Моделирование линейной и нелинейной САУ}
    \begin{block}{Общие положения}
        В данном разделе проводится анализ линейной и нелинейной САУ. В Simulink реализуется система управления на 
        крейсерском режиме полета. Крейсерскому режиму полета для самолета-прототипа Concorde соответствуют 
        М=0,982 и Н=17 км. 
    \end{block}
\end{frame}



\begin{frame}{Обратная динамика}{Цели раздела}
    
\end{frame}
\begin{frame}{Выводы}
    \begin{itemize}
    \item Расчёты лётно-технических характеристик показали, что самолёт примерно соответствует на б/ф режиме похожие характеристики 
    как и у обычного дозвукового самолёта 
    \item В результате расчётов манёвренных возможностей самолёта на высоте $H$ = 6 км было выяснено, что самолёт пригоден для выполнения 
    манёвра придельного виража в установившемся ГП.
    \item Также в работе было показанно, что PI-контроллер улучшает робастность динамической системы в продольном канале управления с применением обратной динамики
    \end{itemize}
\end{frame}
\begin{frame}{Благодарность}
\centering
\huge    % Команда меняет шрифт
Спасибо за внимание
\end{frame}

\end{document}
